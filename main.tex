% ============================================================================
% LaTeX Research Proposal Template
% Created by: Aatiz Ghimire
% Institution: Tribhuvan University, Institute of Science and Technology
% Version: 1.0
% Date: December, 2024
% Description: Template for research proposals based on the required structure.
% License: Open for use by students of Tribhuvan University with appropriate credit.
% ============================================================================

\documentclass[12pt, a4paper]{report}

% Packages
\usepackage{graphicx} % For including images
\usepackage{geometry} % For customizing margins
\usepackage{setspace} % For line spacing
\usepackage{times}    % Times New Roman font for PDFLaTeX
\usepackage{parskip}  % Better control over paragraph spacing
\usepackage{titlesec} % For customizing headings
\usepackage{caption}  % For customizing figure and table captions
\usepackage{amsmath}  % For mathematical equations
\usepackage{amssymb}  % For additional math symbols
\usepackage{apacite}  % For APA references

% Page Margins
\geometry{
    a4paper,
    top=2.5cm,
    bottom=2cm,
    left=3.5cm,
    right=2cm
}

% Line Spacing
\setstretch{1.5}  % Set line spacing to 1.5

% Paragraph Formatting
\setlength{\parindent}{1.5em}  % Indentation for paragraphs

% Heading Styles
\titleformat{\chapter}[block]{\bfseries\fontsize{16}{20}\selectfont}{\thechapter.}{0.3em}{}
\titlespacing*{\chapter}{0pt}{0pt}{0pt}
\titleformat{\section}[block]{\bfseries\fontsize{14}{18}\selectfont}{\thesection}{0.3em}{}
\titleformat{\subsection}[block]{\bfseries\fontsize{12}{16}\selectfont}{\thesubsection}{0.3em}{}

% Caption Styles
\captionsetup[figure]{
    font={bf,small},
    labelfont={bf},
    justification=centering,
    labelsep=colon,
    textfont=bf
}
\captionsetup[table]{
    font={bf,small},
    labelfont={bf},
    justification=centering,
    labelsep=colon,
    textfont=bf,
    position=above
}

% Document Starts
\begin{document}

% Title Page
\begin{titlepage}
    \begin{center}
        \includegraphics[width=0.2\textwidth]{assets/logo.eps}\\[0.5cm] % Replace with the actual logo file path
        \vspace{0.5cm}
        
        {\Large \textbf{Tribhuvan University}}\\
        {\Large \textbf{Institute of Science and Technology}}\\[1.0cm]

        {\large \textbf{A Research Proposal on}}\\

        {\Large \textbf{[Title of the Research Proposal]}}\\[1.0cm]

        \textbf{Submitted by}\\
        {\large \textbf{[Student Name]}}\\
        \textbf{[TU Registration ID]}\\[1cm]

        \textbf{Under the Supervision of}\\
        {\large \textbf{[Supervisor Name(s)]}}\\[1cm]

        \textbf{Submitted to}\\
        {\large \textbf{School of Mathematical Sciences}}\\
        {\large \textbf{Kirtipur, Kathmandu, Nepal}}\\[1cm]

        \textbf{Submitted in partial fulfillment of the requirements for the degree of Master in Data Science}\\[2cm]

        \textbf{[Month, Year]}\\
    \end{center}
\end{titlepage}

% Table of Contents
\tableofcontents

% Roman Page Numbering for ToC and Preliminary Sections
\pagenumbering{roman}
\pagestyle{plain}

\clearpage

%%%%%%%%%%%%%%%%%%%%%%% Chapter 1: Introduction %%%%%%%%%%%%%%%%%%%%%%%%%%%%%
\chapter{Introduction}

% Start Page Numbering from Introduction
\pagestyle{plain} % Default style for numbered pages
\pagenumbering{arabic} % Start Arabic numbering from here

Provide background information about the research problem and its significance.


\section{Research Context}
The research explores the scalability and performance of neural architectures across classical, neuromorphic, and quantum models.

\subsection{Research Questions}
What are the comparative advantages and limitations of classical, neuromorphic, and quantum models in terms of performance and scalability?

%%%%%%%%%%%%%%%%%%%%%%% Chapter 2: Problem Statement %%%%%%%%%%%%%%%%%%%%%%%%%%%%%
\chapter{Problem Statement}
Clearly define the research problem and its scope. Use mathematical models if necessary. For example:

\[
    \min_x \|Ax - b\|_2^2
\]
where \(A\) represents the input data matrix and \(b\) is the output vector.

%%%%%%%%%%%%%%%%%%%%%%% Chapter 3: Objectives %%%%%%%%%%%%%%%%%%%%%%%%%%%%%
\chapter{Objectives}
List the specific objectives of the research:
\begin{itemize}
    \item Objective 1: Analyze classical neural networks.
    \item Objective 2: Evaluate neuromorphic systems.
    \item Objective 3: Explore quantum neural architectures.
\end{itemize}

%%%%%%%%%%%%%%%%%%%%%%% Chapter 4: Rationale of the Study %%%%%%%%%%%%%%%%%%%%%%%%%%%%%
\chapter{Rationale of the Study}
Explain the significance of the research and its potential contributions to the field.

%%%%%%%%%%%%%%%%%%%%%%% Chapter 5: Preliminary Literature Review %%%%%%%%%%%%%%%%%%%%%%%%%%%%%
\chapter{Preliminary Literature Review}
Summarize existing literature related to the topic and identify knowledge gaps.

\begin{figure}[h!]
    \centering
    \includegraphics[width=0.7\textwidth]{example-image-a} % Replace with your image
    \caption{Trends in Neural Network Research}
    \label{fig:neural_trends}
\end{figure}

%%%%%%%%%%%%%%%%%%%%%%% Chapter 6: Methodology %%%%%%%%%%%%%%%%%%%%%%%%%%%%%
\chapter{Methodology}
Describe the research methods and tools. Include any relevant equations or algorithms.

\section{Mathematical Representation}
For example, the gradient descent algorithm is given by:
\begin{equation}
    x_{k+1} = x_k - \alpha \nabla f(x_k)
    \label{eq:gradient_descent}
\end{equation}

%%%%%%%%%%%%%%%%%%%%%%% Chapter 7: Expected Outcomes %%%%%%%%%%%%%%%%%%%%%%%%%%%%%
\chapter{Expected Outcomes}
Outline the anticipated results of the research. For example, Table~\ref{tab:expected_outcomes} summarizes the key outcomes.

\begin{table}[h!]
    \caption{Expected Outcomes}
    \centering
    \begin{tabular}{|c|c|}
        \hline
        Outcome & Description \\ \hline
        Improved Accuracy & Better performance of neural networks. \\ \hline
        Scalability Insights & Insights into hardware-software interactions. \\ \hline
    \end{tabular}
    \label{tab:expected_outcomes}
\end{table}

%%%%%%%%%%%%%%%%%%%%%%% Chapter 8: Working Schedule %%%%%%%%%%%%%%%%%%%%%%%%%%%%%
\chapter{Working Schedule}
Provide a timeline for completing the research. For example:

\begin{table}[h!]
    \caption{Proposed Work Schedule}
    \centering
    \begin{tabular}{|c|c|}
        \hline
        Task & Timeline \\ \hline
        Literature Review & Month 1-2 \\ \hline
        Data Collection & Month 3-5 \\ \hline
        Analysis & Month 6-7 \\ \hline
        Documentation & Month 8-9 \\ \hline
    \end{tabular}
    \label{tab:work_schedule}
\end{table}

%%%%%%%%%%%%%%%%%%%%%%% References %%%%%%%%%%%%%%%%%%%%%%%%%%%%%
\bibliographystyle{apacite}
\bibliography{references} % Replace with the actual .bib file path

\end{document}